\documentclass[a4paper,11pt,twoside]{book}			
\usepackage[latin1]{inputenc}                             
\usepackage[italian]{babel}
\usepackage{lipsum}								
\usepackage{listings}							
\usepackage{url}								
\usepackage{graphicx}							
\usepackage{geometry}						
\usepackage[dvipsnames]{xcolor} 				
\usepackage[hidelinks]{hyperref}		
\usepackage{chngpage}
\usepackage{multicol}

\usepackage{authblk}
\renewcommand\Authand{ e }
		
\geometry{a4paper,top=2cm,bottom=2cm,left=3cm,right=3cm,heightrounded,bindingoffset=10mm}
\raggedbottom									

\lstset{language=Java,						
	showspaces=false,
	showtabs=false,
	breaklines=true,
	showstringspaces=false,
	breakatwhitespace=true,
	commentstyle=\color{ForestGreen},
	keywordstyle=\color{blue},
	stringstyle=\color{red},
	identifierstyle=\color{Gray},
	basicstyle=\small\ttfamily
}

\usepackage{subfiles} % Best loaded last in the preamble

\begin{document}

\title{Interazione Uomo Macchina per Informatici}
\author[1]{Daniele Mazzei, Dipartimento di Informatica Università di Pisa}
\date{a.a. 2020/2021}
\maketitle

\tableofcontents

\subfile{capitoli/capitolo_1.tex} %introduzione
\subfile{capitoli/capitolo_2.tex} %Progettare l'Interazione fra Uomo e Macchina
\subfile{capitoli/capitolo_3.tex} %Human Centered Design
\subfile{capitoli/capitolo_6.tex} %Progettazione delle interfacce
\subfile{capitoli/capitolo_7.tex} %Principi Fondamentali dell'Interazione
\subfile{capitoli/capitolo_8.tex} %Constraints, Discoverability e Feedback
%\subfile{capitoli/capitolo_9.tex} %How do people do things
%\subfile{capitoli/capitolo_4.tex} %Metodi e Strumenti per l'Innovazione
%\subfile{capitoli/capitolo_5.tex} %Le interfacce utente
%\subfile{capitoli/capitolo_10.tex}
%\subfile{capitoli/capitolo_11.tex}
%\subfile{capitoli/capitolo_12.tex}
%\subfile{capitoli/capitolo_13.tex}
%\subfile{capitoli/capitolo_14.tex}




\end{document}
